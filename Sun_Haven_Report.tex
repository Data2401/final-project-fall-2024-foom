% Options for packages loaded elsewhere
\PassOptionsToPackage{unicode}{hyperref}
\PassOptionsToPackage{hyphens}{url}
%
\documentclass[
]{article}
\usepackage{amsmath,amssymb}
\usepackage{iftex}
\ifPDFTeX
  \usepackage[T1]{fontenc}
  \usepackage[utf8]{inputenc}
  \usepackage{textcomp} % provide euro and other symbols
\else % if luatex or xetex
  \usepackage{unicode-math} % this also loads fontspec
  \defaultfontfeatures{Scale=MatchLowercase}
  \defaultfontfeatures[\rmfamily]{Ligatures=TeX,Scale=1}
\fi
\usepackage{lmodern}
\ifPDFTeX\else
  % xetex/luatex font selection
\fi
% Use upquote if available, for straight quotes in verbatim environments
\IfFileExists{upquote.sty}{\usepackage{upquote}}{}
\IfFileExists{microtype.sty}{% use microtype if available
  \usepackage[]{microtype}
  \UseMicrotypeSet[protrusion]{basicmath} % disable protrusion for tt fonts
}{}
\makeatletter
\@ifundefined{KOMAClassName}{% if non-KOMA class
  \IfFileExists{parskip.sty}{%
    \usepackage{parskip}
  }{% else
    \setlength{\parindent}{0pt}
    \setlength{\parskip}{6pt plus 2pt minus 1pt}}
}{% if KOMA class
  \KOMAoptions{parskip=half}}
\makeatother
\usepackage{xcolor}
\usepackage[margin=1in]{geometry}
\usepackage{color}
\usepackage{fancyvrb}
\newcommand{\VerbBar}{|}
\newcommand{\VERB}{\Verb[commandchars=\\\{\}]}
\DefineVerbatimEnvironment{Highlighting}{Verbatim}{commandchars=\\\{\}}
% Add ',fontsize=\small' for more characters per line
\usepackage{framed}
\definecolor{shadecolor}{RGB}{248,248,248}
\newenvironment{Shaded}{\begin{snugshade}}{\end{snugshade}}
\newcommand{\AlertTok}[1]{\textcolor[rgb]{0.94,0.16,0.16}{#1}}
\newcommand{\AnnotationTok}[1]{\textcolor[rgb]{0.56,0.35,0.01}{\textbf{\textit{#1}}}}
\newcommand{\AttributeTok}[1]{\textcolor[rgb]{0.13,0.29,0.53}{#1}}
\newcommand{\BaseNTok}[1]{\textcolor[rgb]{0.00,0.00,0.81}{#1}}
\newcommand{\BuiltInTok}[1]{#1}
\newcommand{\CharTok}[1]{\textcolor[rgb]{0.31,0.60,0.02}{#1}}
\newcommand{\CommentTok}[1]{\textcolor[rgb]{0.56,0.35,0.01}{\textit{#1}}}
\newcommand{\CommentVarTok}[1]{\textcolor[rgb]{0.56,0.35,0.01}{\textbf{\textit{#1}}}}
\newcommand{\ConstantTok}[1]{\textcolor[rgb]{0.56,0.35,0.01}{#1}}
\newcommand{\ControlFlowTok}[1]{\textcolor[rgb]{0.13,0.29,0.53}{\textbf{#1}}}
\newcommand{\DataTypeTok}[1]{\textcolor[rgb]{0.13,0.29,0.53}{#1}}
\newcommand{\DecValTok}[1]{\textcolor[rgb]{0.00,0.00,0.81}{#1}}
\newcommand{\DocumentationTok}[1]{\textcolor[rgb]{0.56,0.35,0.01}{\textbf{\textit{#1}}}}
\newcommand{\ErrorTok}[1]{\textcolor[rgb]{0.64,0.00,0.00}{\textbf{#1}}}
\newcommand{\ExtensionTok}[1]{#1}
\newcommand{\FloatTok}[1]{\textcolor[rgb]{0.00,0.00,0.81}{#1}}
\newcommand{\FunctionTok}[1]{\textcolor[rgb]{0.13,0.29,0.53}{\textbf{#1}}}
\newcommand{\ImportTok}[1]{#1}
\newcommand{\InformationTok}[1]{\textcolor[rgb]{0.56,0.35,0.01}{\textbf{\textit{#1}}}}
\newcommand{\KeywordTok}[1]{\textcolor[rgb]{0.13,0.29,0.53}{\textbf{#1}}}
\newcommand{\NormalTok}[1]{#1}
\newcommand{\OperatorTok}[1]{\textcolor[rgb]{0.81,0.36,0.00}{\textbf{#1}}}
\newcommand{\OtherTok}[1]{\textcolor[rgb]{0.56,0.35,0.01}{#1}}
\newcommand{\PreprocessorTok}[1]{\textcolor[rgb]{0.56,0.35,0.01}{\textit{#1}}}
\newcommand{\RegionMarkerTok}[1]{#1}
\newcommand{\SpecialCharTok}[1]{\textcolor[rgb]{0.81,0.36,0.00}{\textbf{#1}}}
\newcommand{\SpecialStringTok}[1]{\textcolor[rgb]{0.31,0.60,0.02}{#1}}
\newcommand{\StringTok}[1]{\textcolor[rgb]{0.31,0.60,0.02}{#1}}
\newcommand{\VariableTok}[1]{\textcolor[rgb]{0.00,0.00,0.00}{#1}}
\newcommand{\VerbatimStringTok}[1]{\textcolor[rgb]{0.31,0.60,0.02}{#1}}
\newcommand{\WarningTok}[1]{\textcolor[rgb]{0.56,0.35,0.01}{\textbf{\textit{#1}}}}
\usepackage{longtable,booktabs,array}
\usepackage{calc} % for calculating minipage widths
% Correct order of tables after \paragraph or \subparagraph
\usepackage{etoolbox}
\makeatletter
\patchcmd\longtable{\par}{\if@noskipsec\mbox{}\fi\par}{}{}
\makeatother
% Allow footnotes in longtable head/foot
\IfFileExists{footnotehyper.sty}{\usepackage{footnotehyper}}{\usepackage{footnote}}
\makesavenoteenv{longtable}
\usepackage{graphicx}
\makeatletter
\def\maxwidth{\ifdim\Gin@nat@width>\linewidth\linewidth\else\Gin@nat@width\fi}
\def\maxheight{\ifdim\Gin@nat@height>\textheight\textheight\else\Gin@nat@height\fi}
\makeatother
% Scale images if necessary, so that they will not overflow the page
% margins by default, and it is still possible to overwrite the defaults
% using explicit options in \includegraphics[width, height, ...]{}
\setkeys{Gin}{width=\maxwidth,height=\maxheight,keepaspectratio}
% Set default figure placement to htbp
\makeatletter
\def\fps@figure{htbp}
\makeatother
\setlength{\emergencystretch}{3em} % prevent overfull lines
\providecommand{\tightlist}{%
  \setlength{\itemsep}{0pt}\setlength{\parskip}{0pt}}
\setcounter{secnumdepth}{-\maxdimen} % remove section numbering
\usepackage{booktabs}
\usepackage{longtable}
\usepackage{array}
\usepackage{multirow}
\usepackage{wrapfig}
\usepackage{float}
\usepackage{colortbl}
\usepackage{pdflscape}
\usepackage{tabu}
\usepackage{threeparttable}
\usepackage{threeparttablex}
\usepackage[normalem]{ulem}
\usepackage{makecell}
\usepackage{xcolor}
\ifLuaTeX
  \usepackage{selnolig}  % disable illegal ligatures
\fi
\usepackage{bookmark}
\IfFileExists{xurl.sty}{\usepackage{xurl}}{} % add URL line breaks if available
\urlstyle{same}
\hypersetup{
  pdftitle={Sun Haven: More Mana},
  pdfauthor={Phung Tran},
  hidelinks,
  pdfcreator={LaTeX via pandoc}}

\title{Sun Haven: More Mana}
\author{Phung Tran}
\date{2024-12-02}

\begin{document}
\maketitle

\section{Introduction}\label{introduction}

Sun Haven is a farming rpg similar to more popular titles such as
Stardew Valley and Harvest Moon. Players must balance there time between
farming, crafting, and exploration to improve their farm and player
stats. By permanently boosting your stats, such as mana, players can
optimize there gameplay.

To permanently boost your mana stat, there are many events or random
objects that the player can find. However, the most mainline way to do
this would be to eat. Whether it be foragables or cooked foods, eating
is the most basic way to raise mana, making it important to manage your
resources. Specifically because having more mana let's you cast spells
that make farming/combat and traveling more efficient.

Whether its after the first year of game play or beginning of the game,
having this information will save resources for making more storage
containers and saves space in your home. Just like other games in this
genre, you need supplies and space to create the storage. While the game
might give you three homes/farms to travel through, those same resources
can be used for other crafts such as weapons/armor and furnace
materials.

It can be hard to keep track of what foods are the most ideal to
constantly stack up on. This also means you have to mine more ore and
wood, with wood specifically needing more time to regrow. By looking
through the charts presented, we can get useful knowledge about how to
proceed with resource management to increase mana to optimize game play

\begin{Shaded}
\begin{Highlighting}[]
\CommentTok{\# Loading libraires used}

\FunctionTok{library}\NormalTok{(tidyverse) }\CommentTok{\# Loads R packages for data manipulation and visualization}
\FunctionTok{library}\NormalTok{(ggplot2) }\CommentTok{\# Used to create graphs/charts}
\FunctionTok{library}\NormalTok{(janitor) }\CommentTok{\# Used to help clean data }
\FunctionTok{library}\NormalTok{(wesanderson) }\CommentTok{\# Color palettes from Wes Anderson movies }
\FunctionTok{library}\NormalTok{(extrafont) }\CommentTok{\# Loaded in more fonts }
\FunctionTok{library}\NormalTok{(gridExtra) }\CommentTok{\# Create a 2x2 grid for plots}
\FunctionTok{library}\NormalTok{(grid) }\CommentTok{\# Used to change font for gridExtra}
\FunctionTok{library}\NormalTok{(cowplot) }\CommentTok{\# Clean}
\FunctionTok{library}\NormalTok{(knitr) }\CommentTok{\# For kable, cleaner tables }
\FunctionTok{library}\NormalTok{(kableExtra)}
\end{Highlighting}
\end{Shaded}

\begin{Shaded}
\begin{Highlighting}[]
\CommentTok{\# Loading in data that will be used}

\NormalTok{cooked\_food }\OtherTok{\textless{}{-}} \FunctionTok{read.csv}\NormalTok{(}\StringTok{\textquotesingle{}Sun Haven Cooked Food.csv\textquotesingle{}}\NormalTok{) }\CommentTok{\# All cooked meals in Sun Haven game}

\NormalTok{animal\_product }\OtherTok{\textless{}{-}} \FunctionTok{read.csv}\NormalTok{(}\StringTok{\textquotesingle{}Sun Haven Farm Animal.csv\textquotesingle{}}\NormalTok{) }\CommentTok{\# Ranch animals}

\NormalTok{fish }\OtherTok{\textless{}{-}} \FunctionTok{read.csv}\NormalTok{(}\StringTok{"Sun Haven Fish.csv"}\NormalTok{) }\CommentTok{\# All fish in game}

\NormalTok{foragables }\OtherTok{\textless{}{-}} \FunctionTok{read.csv}\NormalTok{(}\StringTok{\textquotesingle{}Sun Haven Foragables.csv\textquotesingle{}}\NormalTok{) }\CommentTok{\# All foragables in game}

\NormalTok{crops }\OtherTok{\textless{}{-}} \FunctionTok{read.csv}\NormalTok{(}\StringTok{\textquotesingle{}Sun Haven Crops.csv\textquotesingle{}}\NormalTok{)  }\CommentTok{\# All crops in game}
\end{Highlighting}
\end{Shaded}

\section{Data Collection}\label{data-collection}

I collected my own data for this project, making data clean up an
important part of this project. Information was complied from two wikis

\url{https://sun-haven.fandom.com/wiki/Sun_Haven_Wiki}
\url{https://sunhaven.wiki.gg/wiki/Sun_Haven_Wiki}

as well as my own save file

The game is still fairly new, with updates still being pushed. Numeric
columns in the cleaned data were updated in micropatches but wikis had
not been.

While using ``csv\_'' is the best choice, I had first created it with
``csv.''. By doing so, the code to clean up is a bit more bloated than
necessary. Different versions of ``NA'' has been appropiateky replaced
for future codes to be handeled correctly. Not all files were used in
the end, however, they were still kept in to show the amount of
consideration that I had to put in collecting and scraping the data.

\begin{center}\rule{0.5\linewidth}{0.5pt}\end{center}

\section{Data Analysis}\label{data-analysis}

By looking at crops grown in the starting town, Sun Haven, that can only
be grown in the Spring and any season, for first play through of the
game, players can figure out which crops will be more cost efficient.
Items that were categorized as trees and flowers were filtered out. The
reason being, flowers are not used in any recipes and fruits from trees
can be found everywhere for free around town while crops have tp be
grown by the player. While this plot can be used to find crops to grow
to maximize profits, we want to looks at the cheapest seeds we can buy
as a new player, and then, crops that would be best to grow after first
year gameplay.

\includegraphics{Sun_Haven_Report_files/figure-latex/Spring_Crops_in_Sun_Haven-1.pdf}

\includegraphics{Sun_Haven_Report_files/figure-latex/Any_Season_Crops_in_Sun_Haven-1.pdf}

For spring/any seaosn during first time gameplay, the top 3 cheapest
crops are going to be Chocoberry (45 Gold) , Green Beans (200 Gold) ,
and Kiwi Berry (250 Gold).

\includegraphics{Sun_Haven_Report_files/figure-latex/Any_Season_Crops_in_Withergate-1.pdf}

\includegraphics{Sun_Haven_Report_files/figure-latex/Any_Season_Crops_in_NelVari-1.pdf}

After first year gameplay, players can choose crops from all the towns
in the game, which includes Withergate and Nel'Vari, as well as consider
some of the pricier crops available to them from Sun Haven alone shown
in the table below.

\begin{table}
\centering
\caption{\label{tab:top_3_expensive_crops}Top 3 Cheapest Crops in Sun Haven (Excluding Flowers and Trees)}
\centering
\begin{tabular}[t]{c|>{}c|c|c}
\hline
\cellcolor[HTML]{D3D3D3}{\textbf{Name}} & \cellcolor[HTML]{D3D3D3}{\textbf{Sell Price}} & \cellcolor[HTML]{D3D3D3}{\textbf{Season}} & \cellcolor[HTML]{D3D3D3}{\textbf{Town}}\\
\hline
Clover & \textcolor{blue}{\textbf{5}} & Any & Sun Haven\\
\hline
Chocoberry & \textcolor{blue}{\textbf{12}} & Spring & Sun Haven\\
\hline
Green Beans & \textcolor{blue}{\textbf{20}} & Spring & Sun Haven\\
\hline
\end{tabular}
\end{table}

Notice the currency for the two other towns are different from the
starter town. Later on, we will be using this information to filter down
the cooked food items we want to focus on.

\begin{center}\rule{0.5\linewidth}{0.5pt}\end{center}

Next, we can look at all the stations that is used to cook food.

\includegraphics{Sun_Haven_Report_files/figure-latex/Station_Recipes_in_each_one-1.pdf}

From Sushi table to cooking pot is all the stations available to you in
the starting town. You can tell that one of the starting station,
cooking pot, has the most cooked food that are available to players at
the beginning of the game.

\includegraphics{Sun_Haven_Report_files/figure-latex/Cooked_Food_Mana_Stat_Boost-1.pdf}

By narrowing it down to only looking at recipes that permanently boost
players mana, it can be seen that the cooking pot still has the most
options available to players. Meaning that whether or not players are
doing their first gameplay, growing crops from Sun Haven is going to be
beneficial in boosting the mana.stat permanently.

\includegraphics{Sun_Haven_Report_files/figure-latex/Amount_of_Mana-1.pdf}

The Cooking Pot has a clear bias towards small stat boosts, which could
indicate that it is designed to provide incremental boosts to players
for first year playthrough. However, there is a noticeable amount of
recipes that offer huge permanent stat boosts, which is geared towards
after first year playthrough.

\begin{center}\rule{0.5\linewidth}{0.5pt}\end{center}

\begin{longtable}[]{@{}
  >{\raggedright\arraybackslash}p{(\columnwidth - 10\tabcolsep) * \real{0.1081}}
  >{\raggedright\arraybackslash}p{(\columnwidth - 10\tabcolsep) * \real{0.2523}}
  >{\raggedright\arraybackslash}p{(\columnwidth - 10\tabcolsep) * \real{0.0811}}
  >{\raggedright\arraybackslash}p{(\columnwidth - 10\tabcolsep) * \real{0.1802}}
  >{\raggedright\arraybackslash}p{(\columnwidth - 10\tabcolsep) * \real{0.1622}}
  >{\raggedright\arraybackslash}p{(\columnwidth - 10\tabcolsep) * \real{0.2162}}@{}}
\caption{Cooking Pot Recipes with Huge Mana Boost}\tabularnewline
\toprule\noalign{}
\begin{minipage}[b]{\linewidth}\raggedright
Station
\end{minipage} & \begin{minipage}[b]{\linewidth}\raggedright
Recipe
\end{minipage} & \begin{minipage}[b]{\linewidth}\raggedright
Currency
\end{minipage} & \begin{minipage}[b]{\linewidth}\raggedright
Ingredient\_1
\end{minipage} & \begin{minipage}[b]{\linewidth}\raggedright
Ingredient\_2
\end{minipage} & \begin{minipage}[b]{\linewidth}\raggedright
Ingredient\_3
\end{minipage} \\
\midrule\noalign{}
\endfirsthead
\toprule\noalign{}
\begin{minipage}[b]{\linewidth}\raggedright
Station
\end{minipage} & \begin{minipage}[b]{\linewidth}\raggedright
Recipe
\end{minipage} & \begin{minipage}[b]{\linewidth}\raggedright
Currency
\end{minipage} & \begin{minipage}[b]{\linewidth}\raggedright
Ingredient\_1
\end{minipage} & \begin{minipage}[b]{\linewidth}\raggedright
Ingredient\_2
\end{minipage} & \begin{minipage}[b]{\linewidth}\raggedright
Ingredient\_3
\end{minipage} \\
\midrule\noalign{}
\endhead
\bottomrule\noalign{}
\endlastfoot
Cooking Pot & Golden Spicy Noodles & Mana Orb & Golden Fire Rune &
Noodles & Greenspice \\
Cooking Pot & Golden Fathom Fruit Stirfry & Mana Orb & Golden Fathom
Fruit & Mushroom & Seaweed \\
Cooking Pot & Golden Mushroom Burger & Mana Orb & Golden Mushroom &
Bread & Tomato \\
Cooking Pot & Golden Brinestone Feast & Mana Orb & Golden Brine Berry &
Golden Trench Nut & Golden Nautitrine Fruit \\
\end{longtable}

These ``huge'' mana stat boost items contain ``Golden'' in the name as
well as ingredients listed that are not possible to obtain early on.
With the currency being ``Mana Orb'', the town Nel'Vari would have to be
available to the player. It can be implied, when ``Golden'' is excluded
from the search, it becomes clear that there are no beginner-friendly
recipes for huge mana stat boosts in the Cooking Pot at the start of the
game.

\begin{center}\rule{0.5\linewidth}{0.5pt}\end{center}

By filtering the search down further, to include only crops from Spring
or Any season players can look at more early game accessible items that
are easier to obtain. This will help significantly with boosting mana
stat efficiently early on so players do not overstock on multiple items
and clutter their farm layout.

The filtered data set shows the top 10 recipes for mana boosts that
include Spring crops as ingredients. It has also been filtered down by
removing any crop whose currency is anything other ``Gold'', as it's
going to be the first currency available for players to use for awhile.

\begin{longtable}[]{@{}
  >{\raggedright\arraybackslash}p{(\columnwidth - 12\tabcolsep) * \real{0.1304}}
  >{\raggedright\arraybackslash}p{(\columnwidth - 12\tabcolsep) * \real{0.2283}}
  >{\raggedright\arraybackslash}p{(\columnwidth - 12\tabcolsep) * \real{0.0978}}
  >{\raggedright\arraybackslash}p{(\columnwidth - 12\tabcolsep) * \real{0.1196}}
  >{\raggedright\arraybackslash}p{(\columnwidth - 12\tabcolsep) * \real{0.1413}}
  >{\raggedright\arraybackslash}p{(\columnwidth - 12\tabcolsep) * \real{0.1413}}
  >{\raggedright\arraybackslash}p{(\columnwidth - 12\tabcolsep) * \real{0.1413}}@{}}
\caption{Top 10 Mana Boosting Foods in Sun Haven (Gold
Currency)}\tabularnewline
\toprule\noalign{}
\begin{minipage}[b]{\linewidth}\raggedright
Station
\end{minipage} & \begin{minipage}[b]{\linewidth}\raggedright
Recipe
\end{minipage} & \begin{minipage}[b]{\linewidth}\raggedright
Currency
\end{minipage} & \begin{minipage}[b]{\linewidth}\raggedright
Boost
\end{minipage} & \begin{minipage}[b]{\linewidth}\raggedright
Ingredient\_1
\end{minipage} & \begin{minipage}[b]{\linewidth}\raggedright
Ingredient\_2
\end{minipage} & \begin{minipage}[b]{\linewidth}\raggedright
Ingredient\_3
\end{minipage} \\
\midrule\noalign{}
\endfirsthead
\toprule\noalign{}
\begin{minipage}[b]{\linewidth}\raggedright
Station
\end{minipage} & \begin{minipage}[b]{\linewidth}\raggedright
Recipe
\end{minipage} & \begin{minipage}[b]{\linewidth}\raggedright
Currency
\end{minipage} & \begin{minipage}[b]{\linewidth}\raggedright
Boost
\end{minipage} & \begin{minipage}[b]{\linewidth}\raggedright
Ingredient\_1
\end{minipage} & \begin{minipage}[b]{\linewidth}\raggedright
Ingredient\_2
\end{minipage} & \begin{minipage}[b]{\linewidth}\raggedright
Ingredient\_3
\end{minipage} \\
\midrule\noalign{}
\endhead
\bottomrule\noalign{}
\endlastfoot
Cooking Pot & Blueberry Salad & Gold & small & Blueberry & Lettuce &
NA \\
Cooking Pot & Pickled Veggie Salad & Gold & small & Lettuce & Tomato &
Greenspice \\
Cooking Pot & Sesame Rice Ball & Gold & small & Rice & Seaweed & NA \\
Cooking Pot & Mochi & Gold & small & Rice & Sugar & NA \\
Cooking Pot & Lasagna & Gold & moderate & Noodles & Tomato & Cheese \\
Cooking Pot & Apple Sauce & Gold & small & Apple & Sugar & Cinnaberry \\
Cooking Pot & Cinnamon Apple Pie & Gold & small & Cinnaberry & Apple &
Sugar \\
Cooking Pot & BLT & Gold & small & Bread & Lettuce & Tomato \\
Cooking Pot & Poke Bowl & Gold & small & Salmon & Tuna & Rice \\
Cooking Pot & Churros & Gold & very small & Cinnaberry & Flour &
Sugar \\
\end{longtable}

For new players starting in Sun Haven, filtering recipes by the Sun
Haven (starting town) and Spring season gives players insight into what
is accessible to them to permanently increase their mana stat. With a
more filtered and focused data, this helps players manage resources
better and help stat progression, while also optimizing their gameplay

As players get pass their first year, a more generalized search of
recipes can help them optimize resource management. By considering all
available recipes, players can find the data useful towards long term
goals and make better decisions in how they would like to use their time
and resources.

In both scenarios, players can use the information to better organize
their farm and storage layouts. This strategic planning allows for a
more efficient gameplay, allowing players to use their increased
permanent mana to cast spells to maximize their crops to recipe yield,
as well as profits and usage of attack spells to get through the
fighting aspects of the game.

\end{document}
